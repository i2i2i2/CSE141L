\documentclass{article}
\usepackage[top=1in, bottom=1in, left=1in, right=1in]{geometry}
\usepackage{amsfonts, amsmath, amssymb, amsthm}
\usepackage{mathtools}
\usepackage{enumerate}
\usepackage{enumitem}
\usepackage{fancyhdr}
\usepackage{listings}
\usepackage{array}
\usepackage{changepage}
\usepackage{longtable}

\lstset{basicstyle=\ttfamily, breaklines=false}
\newcolumntype{L}[1]{>{\raggedright\let\newline\\\arraybackslash\hspace{0pt}}m{#1}}
\newcolumntype{C}[1]{>{\centering\let\newline\\\arraybackslash\hspace{0pt}}m{#1}}
\newcolumntype{R}[1]{>{\raggedleft\let\newline\\\arraybackslash\hspace{0pt}}m{#1}}

\pagestyle{fancy}
\fancyhead[L]{CSE 141L, Computer Architecture}
\fancyhead[R]{Lab1, ISA Design}
\setlength{\headheight}{26pt}
\setlength{\headsep}{16pt}

\title{\quad
  \\ [4.0cm]
  \normalsize \textsc{CSE 141L \\ Spring, 2017}
  \\ [2.0cm]
  \huge \textbf{\uppercase{ISA Design}}
  \\ [0.5cm]
  \normalsize \today}
\author{}
\date{}
\begin{document}
  \maketitle
  \vspace{6.0cm}
  \begin{large}
    \begin{center}
      \begin{tabular}{l r}
        Author:        & Chenxu Jiang \\
        PID:            & A92114155 \\
        Author:        & Dingcheng Hu \\
        PID:            & A92090168 
      \end{tabular}
    \end{center}
  \end{large}
  \newpage
  \section{Introduction}
    \qquad Our achitecture name is Hexac. Instead of single accumulator, which accumulator
    achitecture usually use, we use six accumulators to save the step of load to or
    store from the accumulator. Using six accumulators means accumulator instructions
    are six times of the normal accumulator instructions. We need to compensate the
    need for the large number of instructions by using only 4 accessible temporal
    registers. Small number of accessible registers saves us a lot of space for opcode
    and allows us to invent very specific and bizarre operations for the specified
    3 tasks. Dynamic clock counts are reduced by six accumulators which save step
    for load/store and very specific operations for assigned tasks.
  \section{Instruction Format}
    \begin{adjustwidth}{0.8cm}{}
      \begin{description}
        \item[Set Type] Set 8-bit immediate constant to \texttt{\$t0} \hfill
        \begin{tabular}{|C{0.4cm}|C{3.2cm}|}
          \hline
            0 & 8 bit immediate \\
          \hline
        \end{tabular}
        \item[ex.] \quad\texttt{set\qquad255}\hfill
        \begin{tabular}{|C{0.4cm}|C{3.2cm}|}
          \hline
            0 & 1\ 1\ 1\ 1\ 1\ 1\ 1\ 1\ 1\ 1 \\
          \hline
        \end{tabular}\\
        \item[1-Arg Reg] Operation that takes 2-bit register as argument \hfill
        \begin{tabular}{|C{0.4cm}|C{2.18cm}|C{0.6cm}|}
          \hline
            1 & opcode & reg \\
          \hline
        \end{tabular}
        \item[ex.] \quad\texttt{ldm0\quad\ \$t0\quad\ !\ load memory at \$t0 to \$acc0}\hfill
        \begin{tabular}{|C{0.4cm}|C{2.18cm}|C{0.6cm}|}
          \hline
            1 & 0\ 0\ 0\ 0\ 0\ 0 & 0\ 0 \\
          \hline
        \end{tabular} \\
        \item[Branch] Read relative jump address store at 2-bit lookup table \hfill
        \begin{tabular}{|C{0.4cm}|C{2.18cm}|C{0.6cm}|}
          \hline
            1 & opcode & LT \\
          \hline
        \end{tabular}
        \item[ex.] \quad\texttt{bnzr\quad\ loop\ \ !\ Jump to loop if \$t0 not 0}\hfill
        \begin{tabular}{|C{0.4cm}|C{2.18cm}|C{0.6cm}|}
          \hline
            1 & 0\ 1\ 1\ 1\ 1\ 0 & 0\ 0 \\
          \hline
        \end{tabular} \\
        \item[Miscellaneous] Weird specified operations that takes no arg\hfill
        \begin{tabular}{|C{0.4cm}|C{2.18cm}|C{0.6cm}|}
          \hline
            1 & opcode & 00 \\
          \hline
        \end{tabular}
        \item[ex.] \quad\texttt{atos\_2\qquad\quad\ !\ swap add and sub if \$acc2 negative}\hfill
        \begin{tabular}{|C{0.4cm}|C{2.18cm}|C{0.6cm}|}
          \hline
            1 & 1\ 0\ 1\ 1\ 0\ 0 & 00 \\
          \hline
        \end{tabular}
      \end{description}
    \end{adjustwidth}
  \section{Operations}
    \vspace{-0.4cm}\ttfamily
    \begin{longtable}[c]{L{2.6cm}|L{9.5cm}|C{1.2cm}|C{1.2cm}}
      \multicolumn{4}{ c }{Operation Table}\\
      \hline
      \texttt{Operation} & Description & Format & Opcode\\
      \hline
      \endfirsthead

      \multicolumn{4}{c}{Operation Table Continued}\\
      \hline
      Operation & Description & Format & Opcode\\
      \hline
      \endhead
      \endfoot
      \endlastfoot
      set\ \ \ \ \ [imm] & Set 8-bit unsigned immediate value to \$t0         & Set       & None \\
                         &                                                    &           &      \\
      \hline
      ldm0\ \ \ \ [reg]  & load word in memory location [reg] to \$acc0       & 1-Arg     & 0x00 \\
      ldm1\ \ \ \ [reg]  & load word in memory location [reg] to \$acc1       & 1-Arg     & 0x01 \\
      ldm2\ \ \ \ [reg]  & load word in memory location [reg] to \$acc2       & 1-Arg     & 0x02 \\
      ldm3\ \ \ \ [reg]  & load word in memory location [reg] to \$acc3       & 1-Arg     & 0x03 \\
      ldm4\ \ \ \ [reg]  & load word in memory location [reg] to \$acc4       & 1-Arg     & 0x04 \\
      ldr1\ \ \ \ [reg]  & load word in [reg] to \$acc1                       & 1-Arg     & 0x05 \\
      ldr4\ \ \ \ [reg]  & load word in [reg] to \$acc4                       & 1-Arg     & 0x06 \\
      ldr5\ \ \ \ [reg]  & load word in [reg] to \$acc5                       & 1-Arg     & 0x07 \\
                         &                                                    &           &      \\
      add1\ \ \ \ [reg]  & add word in [reg] to \$acc1, set carry to \$flag   & 1-Arg     & 0x08 \\
      add4\ \ \ \ [reg]  & add word in [reg] to \$acc4, set carry to \$flag   & 1-Arg     & 0x09 \\
      add5\ \ \ \ [reg]  & add word in [reg] to \$acc5, set carry to \$flag   & 1-Arg     & 0x0a \\
      addf0\ \ \ [reg]   & add word in [reg] to \$acc0, \$flag as init carry  & 1-Arg     & 0x0b \\
      addf4\ \ \ [reg]   & add word in [reg] to \$acc4, \$flag as init carry  & 1-Arg     & 0x0c \\
                         &                                                    &           &      \\
      sub1\ \ \ \ [reg]  & sub word in [reg] from \$acc1, set carry to \$flag & 1-Arg     & 0x0d \\
      sub3\ \ \ \ [reg]  & sub word in [reg] from \$acc3, set carry to \$flag & 1-Arg     & 0x0e \\
      sub4\ \ \ \ [reg]  & sub word in [reg] from \$acc4, set carry to \$flag & 1-Arg     & 0x0f \\
      sub5\ \ \ \ [reg]  & sub word in [reg] from \$acc5, set carry to \$flag & 1-Arg     & 0x10 \\
      subf2\ \ \ [reg]   & sub word in [reg] from \$acc2, \$flag as init carry& 1-Arg     & 0x11 \\
      str1\ \ \ \ [reg]  & store word in \$acc1 to [reg]                      & 1-Arg     & 0x12 \\
      str2\ \ \ \ [reg]  & store word in \$acc2 to [reg]                      & 1-Arg     & 0x13 \\
      str3\ \ \ \ [reg]  & store word in \$acc3 to [reg]                      & 1-Arg     & 0x14 \\
      stm0\ \ \ \ [reg]  & store word in \$acc0 to memory location [reg]      & 1-Arg     & 0x15 \\
      stm1\ \ \ \ [reg]  & store word in \$acc1 to memory location [reg]      & 1-Arg     & 0x16 \\
      stm3\ \ \ \ [reg]  & store word in \$acc3 to memory location [reg]      & 1-Arg     & 0x17 \\
      stm4\ \ \ \ [reg]  & store word in \$acc4 to memory location [reg]      & 1-Arg     & 0x18 \\
      stm5\ \ \ \ [reg]  & store word in \$acc5 to memory location [reg]      & 1-Arg     & 0x19 \\
                         &                                                    &           &      \\
      mov\ \ \ \ \ [reg] & copy word in \$t0 to [reg]                         & 1-Arg     & 0x1a \\
      srl\ \ \ \ \ [reg] & logical rshift [reg], set \$flag to shift out      & 1-Arg     & 0x1b \\
      sra\ \ \ \ \ [reg] & arithmetical rshift [reg], set \$flag to shift out & 1-Arg     & 0x1c \\
      srf\ \ \ \ \ [reg] & right shift [reg], set shift in as \$flag          & 1-Arg     & 0x1d \\
                         &                                                    &           &      \\
      \hline
      bnzr\ \ \ \ [addr] & branch if \$t0 not 0, jump to lookup table [addr]  & Branch    & 0x1e \\
      bnuzr\ \ \ [addr]  & branch if upper 4 bits of \$t0 are not 0           & Branch    & 0x1f \\
      b1ne31\ \ [addr]   & branch if \$acc1 is not equal to 31                & Branch    & 0x20 \\
                         &                                                    &           &      \\
      \hline
      dec1               & decrease \$acc1 by 1                               & Misc.     & 0x21 \\
                         &                                                    &           &      \\
      zero0              & zero \$acc0                                        & Misc.     & 0x22 \\
      zero2              & zero \$acc2                                        & Misc.     & 0x23 \\
      zero3              & zero \$acc3                                        & Misc.     & 0x24 \\
      zero4              & zero \$acc4                                        & Misc.     & 0x25 \\
                         &                                                    &           &      \\
      srl4               & logical rshift \$acc4, set shift out to \$flag     & Misc.     & 0x26 \\
      sra5               & arithmetical rshift \$acc5, set shift out to \$flag & Misc.    & 0x27 \\
      srf5               & rshift \$acc5, set shift in as \$flag              & Misc.     & 0x28 \\
                         &                                                    &           &      \\
      sll1               & logical lshift \$acc1, set shift out to \$flag     & Misc.     & 0x29 \\
      slf0               & lshift \$acc0, shift in as \$flag, shift out to \$flag & Misc. & 0x2a \\
      slf2               & lshift \$acc2, shift in as \$flag                  & Misc.     & 0x2b \\
                         &                                                    &           &      \\
      atos\_2            & swap add and sub commands if \$acc2 is negative    & Misc.     & 0x2c \\
      add3\_n2           & add 1 to \$acc3, if \$acc2 not negative            & Misc.     & 0x2d \\
      add4\_n2           & add 1 to \$acc4, if \$acc2 not negative            & Misc.     & 0x2e \\
                         &                                                    &           &      \\
      cp0\_5             & copy \$acc0 to \$acc5                              & Misc.     & 0x2f \\
      add25              & add \$acc5 from \$acc2, set carry to \$flag        & Misc.     & 0x30 \\
      sub25              & sub \$acc5 from \$acc2, set carry to \$flag        & Misc.     & 0x31 \\
      subf25             & sub \$acc5 from \$acc2, \$flag as init carry       & Misc.     & 0x32 \\
                         &                                                    &           &      \\
      inc\_af0           & \$acc0 += \$flag \^\ \$flip                        & Misc.     & 0x33 \\
      add0c25            & add 1 to \$acc0 if upper 4bit of \$acc2, \$acc5 match & Misc.  & 0x34 \\
                         &                                                    &           &      \\
      ld4m1              & load word at memory location \$acc1 to \$acc4      & Misc.     & 0x35 \\
                         &                                                    &           &      \\
      halt               & signal stop execution                              & Misc.     & 0x36
    \end{longtable}
    \rmfamily
  \newpage
  \section{Internal Operand}
    \begin{adjustwidth}{0.8cm}{}
      \begin{description}
        \item[\$acc0 - \$acc5] Five accumulator registers, which can do addition/subtraction and load from/store to register or
          memory address in register. Some of the accumulator registers support shifts. \$acc5 is the special one that many other
          accumulator register can directly read its value and do action with special misc. operations. They all can not be use
          as an argument in \textbf{1-Arg} type operations.
        \item[\$flip] One 1-bit register that convert adder/subtracter, set by \texttt{atos\_2}. If it is set, the \texttt{add}
          operations became \texttt{sub} and \texttt{sub} became \texttt{add}, except inc/decrement for counter or specified.
        \item[\$flag] One 1-bit register that stores shift out and overflow carry bit. Shift and add/sub operation without "f" set
          the \$flag and Shift and add/sub operations with "f" read the \$flag.
        \item[\$t0-\$t3] Four accessible common registers, which can be used as argument in \textbf{1-Arg} operations. Can do
          shift and be used to store temporal value. \$t0 is special that it can be set to 8-bit immediate value with \texttt{set}.
      \end{description}
    \end{adjustwidth}
  \section{Control Flow}
    \qquad Only backward branches for for/while loop are supported. The if branches are saved by \texttt{atos\_2}, since in assigned
    programs, the branch is only different on addition and subtraction. Check Operation table for detailed branch operations.
    The address is calculated by absolute address. Since the branch address is stored in loop-up table with 2 bit control.
    There is a limitation of one program can only have branches at most to 4 different address. There is no limitation on
    maximum branch distance.
  \section{Addressing Mode}
    \qquad Only absolute addressing is supported. The target's absolute address needs to be calculated and stores into temporal registers.
    Then accumulator registers would load from/store to the memory address stored in temporal registers. \\ \\
    \text{\qquad}Ex. Read memory location 128:
    \begin{lstlisting}
      set       128          ! set 128 to $t0
      ldm0      $t0          ! load from memory addr in $t0 to $acc0
      stm0      $t0          ! store to memory addr in $t0 from $acc0
    \end{lstlisting}
  \section{Main Memory}
    \qquad Since the largest given memory is 130, we need main memory to have width of 8-bit and depth of 256.
    Memory is not used a lot. Most operation is done in registers. Only parameter and return result need memory access.
  \section{Dynamic Instruction Count Optimization}
    \begin{itemize}
      \item[1. ] We use six accumulator registers instead of one. This saves a lot of loads and stores of temporal values.
      \item[2. ] We use very specific branch condition, that all branches are calculated in one step instead of two. And we have
        \texttt{atos\_2} instead of if branch, this have 50\% to save a branch always command.
      \item[3. ] We invent a lot of bizarre commands that can only be used in assigned task. The commands may merge 2 commands
        into 1, without lost of too much genericity.
    \end{itemize}
  \section{Cycle Time Optimization}
    \begin{itemize}
      \item[1. ] For every command invented, we verified that the command can be done with at most 3 levels of basic gates, so that no commands are
      too slow compared to the others.
      \item[2. ] Our design forced us to use an adder/subtractor module, with a controller bit as switch. We are forced to implement a
      carry-look-ahead adder instead of ripple adder verilog automatically compiled to.
      \item[3. ] We minimize load and store from and to memory, the memory size is as small as possible which increases speed for load
      and store.
    \end{itemize}
  \section{Competitor Claim}
    \begin{itemize}
      \item[1. ] We have 6 accumulator registers and 4 regular accessibile register. And we can ensure that all operations can be done
        with value stored in registers. Each accumulator registers represents different value. We don't need to load and store different
        value to the same accumulator registers. Our competitors only have 4 accumulator registers. They need to do load and store to memory
        for temporal value and reuse the same register for different value. Therefore, in our design, we use much less load and store.
      \item[2. ] We can load 8-bit immediate value to register in one step. Our competiter need to load immediate value with multiple steps
        Many program would need immediate value. And we save more dynamic counts.
      \item[3. ] We have 6 bits for opcode and can support twices as much operations as our competiters' ISA. With more specific command
        we can do more with less dynamic counts.
    \end{itemize}
  \section{Design with 2 More Bits}
    \qquad We would expose accumulator registers. Now accumulator registers cannot be used as arguments. With 2 more bits we are able to do that
    and we can then have less contrived self-invented instructions. And our ISA can be more generic and supports more programs.
    Now, we use contrived self-invented instructions for optimization of the only three assigned program.
  \section{Machine Classification}
    \qquad Hexac is an accumulator machine.
  \section{Example of Machine Code}
    \qquad Check \textbf{Instruction Format}
  \newpage
  \section{Cordic}
  \subsection{Assembly Code}
  \begin{lstlisting}
    set     24
    ldm3    $t0             ! load lsw of y to $acc3
    set     16
    ldm2    $t0             ! load msw of y to $acc2
    set     8
    ldm1    $t0             ! load lsw of x to $acc1
    set     0
    ldm0    $t0             ! load msw of x to $acc0
    ldr4    $t0             ! set msw of angle to 0 at $acc4
                            ! lsw of angle set to $acc5 later

                            ! prep loop
    cp0_5                   ! store msw of x to $acc5
    str1    $t1             ! store lsw of x to $t1
    str2    $t2             ! store msw of y to $t2
    str3    $t3             ! store lsw of y to $t3
    set     64              ! set $t0 to 0100 0000         (14)

  loop-msw:
    sra5                    ! shift x msw, set shift-out to $flag
    srf     $t1             ! shift x lsw, set shift-in to $flag

    sra     $t2             ! shift y msw, set shift-out to $flag
    srf     $t3             ! shift y lsw, set shift-in to $flag

    atos_2                  ! set adder to subtracter if $acc2 negative

    add1    $t3
    addf0   $t2             ! x new = x + (y >> 1) or x new = x - (y >> 1)

    sub3    $t1
    subf25                  ! y new = y - (x >> 1) or y new = y + (x >> 1)

    add4    $t0             ! sub if flipped

    cp0_5                   ! store new msw of x to $acc5
    str1    $t1             ! store new lsw of x to $t1
    str2    $t2             ! store new msw of y to $t2
    str3    $t3             ! store new lsw of y to $t3

    srl     $t0             ! shift t0
    bnzr    loop-msw        ! backward                     (16*7)

  end-loop-msw:             ! $t1 now x >> 8, $t3 now y >> 8
    sra     $t2             ! shift sign bit of y to $flag
    srl     $t1             ! x always positive, always shift in 0
    srf     $t3             ! shift in sign bit for the first loop
    mov     $t2             ! set $t0 = 2 to $t2, SOURCE of ZERO
    ldr5    $t0             ! set $acc5 to 0
    set     128             ! $t0 = 1000 0000              (6)

  loop-lsw:
    atos_2                  ! set adder to subtracter if $acc2 negative

    add1    $t3             ! x new = x + (y >> 1) or x new = x - (y >> 1)
    inc_af_0                ! $flip? ($flag? do nothing: add 1) : (add $flag)

    sub3    $t1             ! y new = y - (x >> 1) or y new = y + (x >> 1)
    subf2   $t2             ! sub 0 and flag bit

    add5    $t0
    addf4   $t2             ! add to angle or subtract to angle if flipped

    str1    $t1             ! store lsw of x
    srl     $t1             ! shift logical, x always positive
    str3    $t3             ! store lsw of y
    sra     $t3             ! shift arithmetically, y can be negative

    srl     $t0             ! shift right angle
    bnuzr   loop-lsw        ! loop till $t0 = 0000 1000    (13*4)

  end-loop:
    set     5
    stm0    $t0
    set     6
    stm1    $t0
    set     7
    stm4    $t0
    set     8
    stm5    $t0             !                              (8)

    halt                    ! store and stop         Total (193)
  \end{lstlisting}
  \subsection{Dynamic Instruction Count}
  \qquad The instruction count would never vary, since there is no if branch that have different
  instruction counts for different branches. The dynamic instruction count for all executions is 193.
  \section{String Match}
  \subsection{Assembly Code}
  \begin{lstlisting}
    set     94
    ldr1    $t0           ! load 95 to $acc1, counter
    set     95
    ldm5    $t0           ! load last word to $acc5
    set     9
    ldm2    $t0           ! load word to match
    zero0                 ! zero $acc0, pattern match counter   (7)

  loop:
    ld4m1                 ! load word at mem of counter number
    add0c25               ! add 1 to $acc0 if lower 4 bit of $acc2 $acc4 match
    srl4                  ! shift right $acc4
    srf5                  ! shift right $acc5, shift in = shift out of $acc4
    add0c25
    srl4
    srf5
    add0c25
    srl4
    srf5
    add0c25
    srl4
    srf5
    add0c25
    srl4
    srf5
    add0c25
    srl4
    srf5
    add0c25
    srl4
    srf5
    add0c25
    srl4
    srf5                  ! shift right $acc4 $acc5 8 times

    dec1                  ! decrement counter
    b1ne31  loop          ! branch if counter reaches 31        (28*63)

  end-loop:
    add0c25               ! add 1 to $acc0 if lower 4 bit of $acc2 $acc4 match
    srf5                  ! shift right $acc5, shift in = shift out of $acc4
    add0c25
    srf5
    add0c25
    srf5
    add0c25               ! check match extra 4 times,          (7)

    set     10
    stm0    $t0           ! store pattern counter               (2)

    halt                                                  Total (1781)
  \end{lstlisting}
  \subsection{Dynamic Instruction Count}
  \qquad The instruction count would never vary, since there is no if branch that have different
  instruction counts for different branches. The dynamic instruction count for all executions is 1781.
  \section{Division}
  \subsection{Assembly Code}
  \begin{lstlisting}
    set     128
    ldm0    $t0           ! $acc0 msw of dividend
    set     129
    ldm1    $t0           ! $acc1 lsw of dividend
    set     130
    ldm5    $t0           ! $acc5 divisor
    zero2                 ! zero $acc2 reminder                 (7)

    sub25                 ! $acc2 -= divisor, must negative
    sll1
    slf0
    slf2                  ! (reminder, msw Dividend, lsw Dividend) << 1
    add25                 ! $acc2 += divisor, since last op make $acc2 negative
    set     64            ! set $t0 = 0100 0000
    add3_n2 $t0           ! if $acc2 non negative and $t0 to msw of quotient
    set     32            ! set $t0 = 0010 0000                 (8)

  loop-msw:
    atos2                 ! transit adder to subtracter if remainder negative
    sll1
    slf0
    slf2                  ! (reminder, msw Dividend, lsw Dividend) << 1
    sub25                 ! if neg, $acc2 += divisor, if pos $acc2 -= divisor
    add3_n2 $t0           ! if $acc2 >=0 add $t0 to msw quotient, despite flipped

    srl     $t0           ! shift $t0
    bnzr    loop-msw      !                                     (8*6)

  end-loop-msw:
    set     128           ! set $t0 = 1000 0000                 (1)

  loop-lsw:
    atos2                 ! transit adder to subtracter if remainder negative
    sll1
    slf0
    slf2                  ! (reminder, msw Dividend, lsw Dividend) << 1
    sub25                 ! if neg, $acc2 += divisor, if pos $acc2 -= divisor
    add4_n2 $t0           ! if $acc2 >=0 and $t0 to lsw quotient, despite flipped

    srl     $t0           ! shift $t0
    bnzr    loop-lsw      !                                     (8*8)

  end-loop-msw:
    set     126
    stm3    $t0           ! store msw of quotient
    set     127
    stm4    $t0           ! lsw of quotient                     (4)

    halt                                                  Total (133)
  \end{lstlisting}
  \subsection{Dynamic Instruction Count}
  \qquad The instruction count would never vary, since there is no if branch that have different
  instruction counts for different branches. The dynamic instruction count for all executions is 133.
\end{document}
